%\section*{Remerciement}
\newpage
\begin{otherlanguage}{arabic}   %\section*{تاريخ}
\begin{abstract}
\justify
\begin{center}
جدولة \textLR{DAG} مع تموضع المعطيات على بنية  \textLR{NUMA}  المتعددة النواة
\end{center}

خورزميات الجدولة القائمة على المخططات الموجهة اللاحلقية  \textLR{DAG} المستعملة في اطار الحوسبة التقليدية تصب جل تركيزها على المهمات الحسابية اكثر من تموضع المعطيات خلال عملية الجدولة و التموضع.
في سياق الذاكرات المتسلسلة الترتيب, هذا النمط يكون له اثر سلبي على اداء و فعالية النظام ككل.
مع استحداث البنى المتعددة النواة و ذات ذاكرة لا متماثلة و غير متناظرة الولوج (\textLR{NUMA}) فان ايجاد الجدولة-الحل و التموضع الامثلين في ان واحد لمسألة جدولة و تموضع مخطط \textLR{DAG} يعد من الامور المستعصية التي تشكل تحديا في ادبيات البحث العلمي الخاص بمسائل الجدولة.
بنى كهذه تعرض ذكرات متسلسلة الترتيب و متعددة المستويات بخصائص مختلفة, الامر الذي يجعل تكلفة الاتصال متباينة في كل مستوى من جهة و من جهة اخرى بعض المهام المترابطة لها جانب اتصلي معتبر غالبا ما تحمل و تحفظ معطياتها من و الى الذاكرة مرارا متعلقة بتموضع هذه الاخيرة.
اداء هذا النظام و أداة الجدولة لا يعتمد فقط على جدولة عمليات التنفيذ المخففة و لكن ايضا على قرار التموضع الخاص بمعطياتها.

في هذه الاطروحة نسعى الى حل مسألة جدولة التطبيقات المتوازية المعبر عنها بـ \textLR{DAG} على البنى المستهدفة و ذلك من خلال استكشاف الحالة التي ليس فقط الحوسبة و الاتصال يأخذان بعين الاعتبار خلال اتخاذ قرار الجدولة و لكن ايضا تموضع المعطيات على ذكرات البنى اللامتناظرة. ان سياسات الجدولة و التموضع الحالية تسعى الى التقليل من حدة التأثير العام لمعوق الولوج البعيد بأخذها بالحسبان تموضع المعطيات في قرار الجدولة. 
   لكن معظم الابحاث السابقة كانت في سياق المهمات المستقلة. لتطبيق ذلك على تنفيذ التطبيقات المعبر عنها بـ \textLR{DAG} على البنى المتسلسلة الترتيب, من الضروري ايجاد سبيل للمزج بين كلتا السياستين للتوصل الى القرار الانسب حول متى و اين  نجدول عمليات التنفيذ و اين نضع معطياتها.
   في هذا العمل, سوف نقوم بـ :
   \begin{enumerate}
\item ادراج معلومة طبولوجيا بنى التنفيذ (المقدمة من بيئة مرحلة التنفيذ خلال بدء التطبيق).
\item استخدام نمط التطبيق(بنية التطبيق معطاة من طرف المبرمج).
\item بناءا على هذه البنية و حالة المهمات, سوف نجزء مجموعة المهمات على عدد من المجموعات المنفصلة.
\item توسيع مجال رؤية المهمات باستكشاف افق اوسع و ذلك لجمع معلومات اكثر حول الحالة الراهنة لعملية تنفيذ \textLR{DAG}.
\item حفظ توازن العبء على البنية باستخدام استراتيجية سرقة العمل  خلال مرحلة التنفيذ القائمة على المسافة 
\end{enumerate}
    
هذه هي الافكار الرئيسية المقترحة لسياسات الجدولة و التموضع لهذا العمل المدرجة كأرستيكس في هذه العملية لارشادها و للتقليل من اثر المعوقات  على زمن الانتهاء الكلي و لحفظ اداء النظام.

\textbf{كلمات-مفتاح} :  
جدولة \textLR{DAG}, تموضع المعطيات, جوارية المعطيات, حواسيب بمعالج متعدد النواة, بنى \textLR{NUMA}, بنى متسلسلة الترتيب.
\justify
\end{abstract}
\end{otherlanguage}